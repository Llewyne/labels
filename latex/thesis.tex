\documentclass[a4paper]{article}

\usepackage{cite}

\title{Labeling Sloped Nonograms Literature}

\author{Eva Timmer}

\date{\today}

\begin{document}
\maketitle

\section{Literature}
\subsection{On (solving) Nonograms}
It is neccesary to describe what nonograms are, specifically the new types of nonograms (curved and sloped) that require automatic labeling. If removing labels is allowed in order to reach a solution, then we must also include an accurate algorithm to determine if the puzzle is still solvable. Otherwise we can assume the initial puzzle and set of descriptions is simple and solvable.\\
\cite{van2017improved}\\
\cite{ortiz2007automated}\\

\subsection{On (solving) puzzles in general}
\cite{salcedo2007solving}\\
\cite{smith2013quantifying}\\

\subsection{On labeling}
Labeling nonograms is similar to labeling graphs. Therefore research on automatically labeling graph boundaries is helpful.\\
\cite{boundaryLabelling}\\
\cite{Bekos:2006:PLM:1151903.1151906}\\
\cite{Gemsa:2011:BAP:2093973.2094012}\\

\subsubsection{Floating labels: Applying dynamic potential fields for label layout \cite{hartmann2004floating}}
10/12 cited 67 times

\subsubsection{Dynamic map labeling \cite{been2006dynamic}}
10/12 cited 123 times

\subsection{On graph/puzzle aesthetics}
At minimum, the goal is to design an algorithm that can determine a labeling where all labels are included and non of the labels overlap, if such a labeling exists. We can apply additional restrictions in order to reach a solution that is also aesthetically pleasing. In order to do this we must first determine what is aesthetically pleasing.\\

\subsubsection{Validating Graph Drawing Aesthetics \cite{purchase1995validating}}
An empirical study on three measurable aesthetic qualities of graphs. These aesthetics are symmetry, minimize edge crossings and minimize bends. Only the first applies to this project, as we already assume that edge crossings and bends will reduce the understandability of a labeling, and therefore will only consider labelings without edge crossings and bends. Unfortunately the results regarding symmetry in this study are inconclusive. However, the authors do provide some ideas on how to perform a conclusive study of the symmetrical aesthetic.\\
12/12 cited 299 times
\subsubsection{Metrics for Graph Drawing Aesthetics \cite{purchase2002metrics}}
This paper presents metrics for measuring the aesthetic presence in a graph drawing for seven common aesthetic criteria. While none of the metrics except "maximizing symmetry" could apply to the problem of labeling, the general ideas concerning how to choose and use are still relevant. The paper proposes that measures should be continuous rather than binary. This is also true for puzzle labeling. Assuming we have a set of measurable aesthetic criteria, we can design an algorithm that minimizes or maximizes these aspects of the labeling.\\
10/12 cited 262 times

\subsubsection{Cognitive measurements of graph aesthetics \cite{ware2002cognitive}}
Focusses more on how the brain interprets visuals. \\
10/12 cited 404 times

\subsubsection{Metrics for functional and aesthetic label layouts \cite{hartmann2005metrics}}
Presents a system for the internal and external labeling of images. \\
10/12 cited 65 times


\subsubsection{Specifying label layout style by example \cite{vollick2007specifying}}
10/12 cited 43 times

\subsubsection{On the readability of boundary labeling \cite{barth2015readability}}
Only focusses on different ways of drawing leaders \textit{inside} the boundary and is therefore not really relevant.

12/12 cited 13 times

\section{Introduction}
\subsection{What is Nonogram}
\subsection{What is sloped nonogram}
\section{Method}
\subsection{Programming language}
SAT solving library
\subsection{Input}
Boundary, what are the labels, pairs of ports, slope per pair, seperate picture (vector)
\subsubsection{Only certain angles (45,isometric)}
\subsubsection{Any set of k angles}
\subsubsection{All angles}
\subsubsection{Curves}
\subsection{Output}
\subsubsection{Image}
\subsubsection{Assignment of labels}
\subsection{Comparison}
\subsubsection{Focus on one aspect}
\subsubsection{Focus on combinations of aspects}
\subsubsection{Grid type}
\subsubsection{Speed}
\subsubsection{Readability/Aesthetics}
\subsubsection{Difficulty}
\newpage
\section{Sub problems / choices}
\subsection{Label size}
\subsubsection{Amount squares}
\subsubsection{Only reference (actual size of label)}
\subsection{Leader length}
\subsubsection{Fixed length}
\subsubsection{Maximum length}
\subsubsection{No restrictions}
\subsection{Sides}
\subsubsection{Top and bottom same side}
\subsubsection{Top and bottom opposite side}
\subsubsection{All top on one side and all bottom on other side}
\subsubsection{Whatever fits}
\subsection{Omitting labels}
\subsubsection{Least amount to fit}
\subsubsection{Remove redundant}
\subsection{Different shape labels}
\section{Special cases}
\subsection{Corners}
\subsection{Concave}

\section{Results}



\bibliography{refs}{}
\bibliographystyle{plain}
\end{document}